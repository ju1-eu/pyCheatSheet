%ju 22-Jul-21 keywords.tex
\section{boolean true false}\label{boolean-true-false}

\lstset{language=Python}% C, TeX, Bash, Python 
\begin{lstlisting}[
	%caption={}, label={code:}%% anpassen
][language=Python]
# boolean true, false
log_var1 = True == (1 > 2) # False
log_var2 = True == (2 > 1) # True
\end{lstlisting}

\section{and or not}\label{and-or-not}

\lstset{language=Python}% C, TeX, Bash, Python 
\begin{lstlisting}[
	%caption={}, label={code:}%% anpassen
][language=Python]
# and, or, not
log_var3 = True and True # True
log_var4 = True or False # True
log_var5 = not False # True
\end{lstlisting}

\section{break continue}\label{break-continue}

\lstset{language=Python}% C, TeX, Bash, Python 
\begin{lstlisting}[
	%caption={}, label={code:}%% anpassen
][language=Python]
# break, continue
while True:
    break    # ende
    continue # abbruch
\end{lstlisting}

\section{class}\label{class}

\lstset{language=Python}% C, TeX, Bash, Python 
\begin{lstlisting}[
	%caption={}, label={code:}%% anpassen
][language=Python]
# class
#class Coffee:
    # Define your class
\end{lstlisting}

\section{funktion}\label{funktion}

\lstset{language=Python}% C, TeX, Bash, Python 
\begin{lstlisting}[
	%caption={}, label={code:}%% anpassen
][language=Python]
def say_hi():
    print("hi")


say_hi()
\end{lstlisting}

\section{if elif else}\label{if-elif-else}

\lstset{language=Python}% C, TeX, Bash, Python 
\begin{lstlisting}[
	%caption={}, label={code:}%% anpassen
][language=Python]
x = int(input("Eingabe Zahl: "))
if x > 3: print("Big")
elif x == 3: print("3")
else: print("Small")
\end{lstlisting}

\section{for while}\label{for-while}

\lstset{language=Python}% C, TeX, Bash, Python 
\begin{lstlisting}[
	%caption={}, label={code:}%% anpassen
][language=Python]
# For 
for i in [0,1,2]:
    print(i)

# While 
j = 0
while j < 3:
    print(j); j = j + 1
\end{lstlisting}

\section{in}\label{in}

\lstset{language=Python}% C, TeX, Bash, Python 
\begin{lstlisting}[
	%caption={}, label={code:}%% anpassen
][language=Python]
# in
liste = [2, 39, 42]
log_var6 = 42 in liste # True
log_var6
\end{lstlisting}

\section{is}\label{is}

\lstset{language=Python}% C, TeX, Bash, Python 
\begin{lstlisting}[
	%caption={}, label={code:}%% anpassen
][language=Python]
# is
y = x = 3
log_var7 = x is y # True
\end{lstlisting}

\section{None}\label{none}

\lstset{language=Python}% C, TeX, Bash, Python 
\begin{lstlisting}[
	%caption={}, label={code:}%% anpassen
][language=Python]
# None
print() is None # True
\end{lstlisting}

\section{lambda}\label{lambda}

\lstset{language=Python}% C, TeX, Bash, Python 
\begin{lstlisting}[
	%caption={}, label={code:}%% anpassen
][language=Python]
# lambda
(lambda x: x+3)(2) # 5
\end{lstlisting}

\section{return}\label{return}

\lstset{language=Python}% C, TeX, Bash, Python 
\begin{lstlisting}[
	%caption={}, label={code:}%% anpassen
][language=Python]
# return
def increment(x):
    return x + 1

    
increment(4) # returns 5
\end{lstlisting}
